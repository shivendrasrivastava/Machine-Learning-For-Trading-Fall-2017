\documentclass[a4paper]{article}

% \usepackage{algpseudocode}
\usepackage{float}
\usepackage[english]{babel}
\usepackage[utf8x]{inputenc}
\usepackage{amsmath}
\usepackage{amssymb}
\usepackage{graphicx}
\usepackage[colorinlistoftodos]{todonotes}
\usepackage{algorithm}
%\usepackage{algorithmic}
\usepackage{algpseudocode}
\usepackage{amsmath}
\usepackage{graphics}
\usepackage{epsfig}
% \usepackage[]{algorithm2e}
\usepackage{enumitem}
% \usepackage{enumerate,url}


\title{CSE 6140 Assignment 3 Solution for Problem 1-2}
\author{Shu-Ho, CHOU}
\begin{document}
\maketitle
%
% \noindent Please upload:\\
%  1) a PDF cover letter indicating with whom you worked (if applicable), the sources you used, and if you wish, your impressions about the assignment (what was fun, what was difficult, why...); \\
%  2) a PDF with your solutions of Problems 1, 2 and 3; \\
%  3) a PDF of your report for Problem 4; \\
%  4) a single zip file of your code, README, results for Problem 4.\\
%  Each file name should start by \url{<GTusername>_HW1}.
% \\\\
% \noindent Please type your answers in \LaTeX. You may handwrite them if you
% wish, but if we cannot read your handwriting, you will not receive points
% for your answer.
% \\\\
% \noindent Note that MSTs (Problem 4) will be discussed in class on Sept. 5 (see
% course schedule on T-Square).
\section{Fiber Connectivity}
First of all, if we want to connect all cities together, we need at least $|C|$ - 1 edges
to get all cities connected, so basically we can state the problem in another way: \\
We want to find a subset T such that $|T| = |C| - 1$ such that $1 \leq d_{T}(u) \leq k$
for all nodes $u \in C$, and all cities in C are connected.\\
\\
To prove it in NP-complete, we first prove it NP. \\
Say we have a candidate solution $S \subseteq R$, then we can take the following steps to
verify whether it's a valid solution. \\
\begin{enumerate}
	\item check whether $|S| = |C| - 1$, if not, it's not valid. And this takes at most $O(n)$
	by counting all elements in S.
	\item check if all cities are connected together by S: simply do BFS or DFS and see if all nodes are
	visited. Takes O(n) time.
	\item check if all nodes have at most k edges connected to it: if the graph is stored by adjacency list,
	by checking the number of nodes each node is connected to, we will visit at most O(n) edges, and so it
	takes at most O(n) time to verify this.
\end{enumerate}
So, it takes polynomial time to verify whether a candidate is a valid solution.\\
\\
Secondly, we want to prove it in NP-hard by reducing a NP-complete problem to it. \\
We show that we can reduce Hamilton Path to this problem. Here we consider the case
when k = 2. And the Hamilton Path problem is that in a graph G = (V, E), we can find
a path that contains every vertex exactly once.\\
To reduce the problem of a HamiltonPath(G) to the specific problem, say, FiberConnect(G', k=2),
we only need to copy all the vertices and edges from G to G' = (V', E'), so the reduction takes $O(|V|+|E|)$,
which is polynomial time.\\
\\
Then we need to prove that \textbf{HamiltonPath(G)} has a yes instance if and only if
\textbf{FiberConnect(G', k=2)} has a yes instance.
\begin{itemize}
 	\item \textbf{HamiltonPath(G)} $\implies$ \textbf{FiberConnect(G', k=2)} \\
		Let $S \subseteq E$ be a solution to \textbf{HamiltonPath(G)}. We claim that
		S is also a solution to \textbf{FiberConnect(G', k=2)}. From the definition of
		Hamilton Path we already know that it connects all the vertices with $|V| - 1$,
		which is $|C| - 1$ edges. For the sake of contradiction, we say that S will not
		be a valid solution to \textbf{FiberConnect(G', k=2)}. If this is true,
		then there exist a vertex u in G' that is connected to at least 3 edges. Therefore,
		the corresponding vertex u in G will be visit at least twice because a vertex can
		connect to a maximum of two edges (one in edge, and one out edge) with only one visit.
		Which is a contradiction to the original assumption that S is a solution for \textbf{HamiltonPath(G)}.
		% Let $S \subseteq V$ be a vertex cover in G with $|S| \leq k$. We claim
		% that S is also a solution to G'.
		% To prove it right, we can discuss all vertices in two cases: \\
		% 1. The vertices that are originally in V, then either they are themselves
		% in S or there are edges that directly connect to some other vertex $u \in S$.
		% So in PYGMALION, those vertices either have a bunker or are directly connected
		% to a town with a bunker. \\
		% 2. For a vertice, say x, that is not in the original V, we know that it
		% is connected two vertices that are originally in V, say u and v. We state
		% that at least one of u or v is in S. We prove it by contradiction by saying
		% that neither u and v is in S. Then we know that the edge (u, v) in E is not
		% covered by S, which is contradict to the original statement that S is a
		% cover set, so at least one of u and v must be in S, and thus x is connected
		% to at least one town with bunker.
	\item \textbf{FiberConnect(G', k=2)} $\implies$ \textbf{HamiltonPath(G)} \\
		Let $S \subseteq E'$ be a solution to \textbf{FiberConnect(G', k=2)} we know that
		$|S| = |C| - 1 = |V| - 1$, and for all nodes in V', the edges connected is at most 2,
		and that all vertices are connected.
		We claim that S is also a solution to \textbf{HamiltonPath(G)}. \\
		For the sake of contradiction, assume it is not a solution to \textbf{HamiltonPath(G)}.
		If this is true, there must be a vertex x connected to at least 3 edges in S, to make
		it visited at least twice. However, we know that each vertex in V' are also in V, and
		so are connected to at most 2 edges in S, which is a contradiction. Hence, S must also
		be a solution to \textbf{HamiltonPath(G)}.
		% Let $S \subseteq V'$ be a solution to \textbf{PYGMALION(G', k')}. First
		% we consider the vertices that are not originally in V, say we have a vertex
		% x that is in V' and not in V, we can see that x is connected exactly to two
		% vertices u, v that are originally in V. And since both u and v can cover more
		% (u-x-v, and uv connected) or equal to that x can, we can replace x by u or v.
		% By doing this with all the nodes that are originally not in V, we get a new solution
		% S, which still has k' = k vertices and is still a valid solution to
		% \textbf{PYGMALION(G', k')}. And we know that for a vertex x in V' but not V
		% is connected to a vertex in S, it means that the edge (u, v), which u, v connected to x
		% will be covered by S. Thus all edges in E will be covered by S. And so if we have
		% any solution D that satisfies \textbf{PYGMALION(G', k')}, it can be transferred to
		% a solution S that satisfies \textbf{vertex-cover(G, k)}.
\end{itemize}

Since it is both NP and NP-hard, so we can say that it's in NP-Complete.



\section{In Hartford, Hereford, and Hampshire}
Say the undirected graph G = (V, E), with E is the set of
roads that connect one town to another, and V is the set of
all the towns. \\
\\
To prove that it is NP-complete, the first step
is to prove it NP.\\
\\
Say we have a candidate solution $C \subseteq V$. To verify whether
C is a solution, we can take the following steps:
\begin{enumerate}
	\item check if the number of vertices $|C|$ is smaller or equals to k, which takes at most $O(k) = O(|V|)$ time
	\item check if all vertices are either in the given set $C$ or one of its edges has an endpoint in
	that set. It can be simply done by marking all nodes that are in set C, and all the nodes that are connected to
	the nodes in C, and see if all nodes are marked. This takes at most $O(|E| + |V|)$ time since we only need to loop
	over each edge and each node.

\end{enumerate}
So, it takes polynomial time to verify whether a candidate is a valid solution.\\
\\
Secondly, we want to prove it in NP-hard by reducing a NP-complete problem to it. \\
We show that we can reduce \textbf{Vertex Cover} to PYGMALION in polynomial time. \\
\textbf{vertex-cover(G, k)} is that whether we can find a set of vertices $S \subseteq V$
such that each edge has at least one node inside this set. \\
Now we reduce this problem to PYGMALION by constructing the input for \textbf{PYGMALION(G', k')}.
Let k' = k, and construct a new G' = (V', E'). Let V' and E' initialized as V and E,
and for each edge (u, v) in E, we add a new vertex x and two vertices (u, x) and (x, v)
to V' and E' respectively. \\
To do the reduction, we need to first copy the original graph ($O(|V| + |E|)$), and then loop over all the
edges to add vertices and edges to V' and E' ($O(E)$), so it takes $O(|V|+|E|)$, which is polynomial time to
reduce the problem.\\
\\
After reducing the problem, we need to prove that \textbf{vertex-cover(G, k)} has a yes instance if and only if
\textbf{PYGMALION(G', k')} has a yes instance. \\
\begin{itemize}
 	\item \textbf{vertex-cover(G, k)} $\implies$ \textbf{PYGMALION(G', k')} \\
		Let $S \subseteq V$ be a vertex cover in G with $|S| \leq k$. We claim
		that S is also a solution to G'.
		To prove it right, we can discuss all vertices in two cases: \\
		1. The vertices that are originally in V, then either they are themselves
		in S or there are edges that directly connect to some other vertex $u \in S$.
		So in PYGMALION, those vertices either have a bunker or are directly connected
		to a town with a bunker. \\
		2. For a vertice, say x, that is not in the original V, we know that it
		is connected two vertices that are originally in V, say u and v. We state
		that at least one of u or v is in S. We prove it by contradiction by saying
		that neither u and v is in S. Then we know that the edge (u, v) in E is not
		covered by S, which is contradict to the original statement that S is a
		cover set, so at least one of u and v must be in S, and thus x is connected
		to at least one town with bunker.
	\item \textbf{PYGMALION(G', k')} $\implies$ \textbf{vertex-cover(G, k)} \\
		Let $S \subseteq V'$ be a solution to \textbf{PYGMALION(G', k')}. First
		we consider the vertices that are not originally in V, say we have a vertex
		x that is in V' and not in V, we can see that x is connected exactly to two
		vertices u, v that are originally in V. And since both u and v can cover more
		(u-x-v, and uv connected) or equal to that x can, we can replace x by u or v.
		By doing this with all the nodes that are originally not in V, we get a new solution
		S, which still has k' = k vertices and is still a valid solution to
		\textbf{PYGMALION(G', k')}. And we know that for a vertex x in V' but not V
		is connected to a vertex in S, it means that the edge (u, v), which u, v connected to x
		will be covered by S. Thus all edges in E will be covered by S. And so if we have
		any solution D that satisfies \textbf{PYGMALION(G', k')}, it can be transferred to
		a solution S that satisfies \textbf{vertex-cover(G, k)}.
\end{itemize}

Since it is both NP and NP-hard, so we can say that it's in NP-Complete.
\end{document}
